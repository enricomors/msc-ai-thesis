\documentclass[a4paper,singleside,12pt]{report} % Uncomment this for single side pdf.
%\documentclass[a4paper,twoside,12pt]{report} % Uncomment this for printing.

\usepackage{ai_bo_thesis}
\usepackage[english]{babel}
\usepackage[T1]{fontenc}
\usepackage{lmodern}
\usepackage{hyperref}  % Fixes \href errors
\usepackage{xurl}      % Allows line breaks in URLs
\usepackage{booktabs}  % Fixes \toprule, \midrule, \bottomrule errors
\setlength{\headheight}{14.5pt}  % Fixes fancyhdr warning

\usepackage{csquotes} % Recommended for proper quotation handling
\usepackage[backend=biber, style=numeric, sorting=nty, firstinits=true]{biblatex}
\addbibresource{biblio.bib}

\setmainfont{Times New Roman}
\begin{document}

\title{Tensor-prolog: a logic programming framework for training neural networks}
\topic{Applied Logic Programming}
\candidate{John Smith}
\supervisor{Prof.~Mario Rossi}
%	\cosupervisor{& John Doe, PhD.} % One co-supervisor.
%	\cosupervisors{& Dott.~Ing.~Luigi Bianchi\\& Dott~Avv.~Lucia Rossi} % More than one co-supervisor.
\academicyear{2020-2021}
\session{1st}

\frontispiece 
\dedication{dedicated(X) :- friend(X).}
\toc
\figstoc
\tablestoc
\begintext

\chapter{Introduction}

\section{Background and Rationale}

The adoption of Artificial Intelligence (AI) is growing rapidly, leading to the development of increasingly 
complex models that demand significant computational resources. This growth results in high carbon emissions, 
raising concerns about sustainability in AI research and deployment. Consequently, there is a need to monitor and 
reduce the environmental impact of AI computations.

In this thesis, we investigate the problem of optimizing hardware architecture selection and configuration for AI 
algorithms while adhering to carbon emission constraints. An existing tool, 
\href{https://www.sciencedirect.com/science/article/abs/pii/S0950705122005974}{HADA} 
(HArdware Dimensioning of AI Algorithms), has previously tackled hardware dimensioning considering budget, runtime, 
and solution quality constraints. Our work extends HADA by integrating carbon emission constraints, introducing the 
concept of Sustainable Hardware Dimensioning. \cite{wielemaker2012swi}

\chapter{Related Works}
\section{Carbon Footprint in AI}
Several studies have addressed the issue of AI’s carbon footprint. Tools like \textbf{Green Algorithms} and 
\textbf{CodeCarbon} have been developed to estimate and monitor emissions.

\textbf{CodeCarbon} is an open-source tool designed to track the energy consumption of computational resources 
and estimate the corresponding carbon emissions. The formula used is:

\begin{equation}
CO2eq = C \times E
\end{equation}

where:
\begin{itemize}
\item \textbf{C} represents the carbon intensity of electricity (kg CO2e per kWh), varying by country and energy mix.
\item \textbf{E} is the total electricity consumed during computation (kWh).
\end{itemize}

By monitoring CPU, GPU, and RAM consumption, CodeCarbon estimates the total emissions associated with a computation. 
It retrieves the carbon intensity based on the geographical location and logs results at user-defined intervals 
(default: 15 seconds).

Installation:
\begin{verbatim}
pip install codecarbon
\end{verbatim}

\chapter{Metodology}

\section{Empirical Model Learning in HADA}

HADA employs the \textbf{Empirical Model Learning (EML)} paradigm, which integrates \textbf{Machine Learning (ML)} models 
into an optimization framework. EML involves:

\begin{enumerate}
\item \textbf{Data Collection}: Running target algorithms under various hyperparameter configurations and hardware settings 
to collect performance data.
\item \textbf{Surrogate Model Creation}: Training ML models (e.g., Decision Trees) to approximate the relationship between 
input configurations and performance metrics (e.g., runtime, memory, carbon emissions).
\item \textbf{Optimization}: Using the learned models within a combinatorial optimization framework to find the best hardware
 configuration.
\end{enumerate}

HADA was originally applied to the \textbf{ANTICIPATE} and \textbf{CONTINGENCY} stochastic algorithms used in energy 
management. These algorithms compute energy production schedules while minimizing cost and considering uncertainties.

\section{Integration of CodeCarbon in HADA}

To extend HADA for sustainable AI, we integrate CodeCarbon to track emissions in:
\begin{itemize}
\item ANTICIPATE and CONTINGENCY algorithms.
\item MaxFlow Algorithms:
\begin{itemize}
\item Boykov-Kolmogorov (BK)
\item Excess Incremental Breadth First Search (EIBFS)
\item Hochbaum's Pseudo Flow (HPF)
\end{itemize}
\end{itemize}

\chapter{Experimental Analysis}

\section{Benchmarking on Different Hardware Platforms}
Experiments were conducted on:
\begin{itemize}
\item MacBook Pro (2019)
\item Leonardo Supercomputer (CINECA HPC)
\end{itemize}

Each algorithm was executed on 30 instances with hyperparameter values ranging from 1 to 100, 
generating datasets with 6,000 records per algorithm per hardware platform.

\chapter{HADA-as-a-Service}

\section{HADA Web Application}
Benchmark data was integrated into the HADA web application, requiring:
\begin{itemize}
\item Creation of JSON configuration files for each algorithm-hardware combination.
\item Specification of hyperparameters and performance targets.
\end{itemize}

Example JSON structure:
\begin{verbatim}
{
"name": "anticipate",
"HW_ID": "macbook",
"hyperparams": [
{"ID": "num_scenarios", "type": "int", "LB": 1, "UB": 100}
],
"targets": [
{"ID": "time", "LB": null, "UB": null},
{"ID": "memory", "LB": null, "UB": null},
{"ID": "emissions", "LB": null, "UB": null}
]
}
\end{verbatim}

\chapter{Conclusions}

This work extends HADA by integrating carbon emission constraints, enhancing its applicability 
for sustainable AI hardware selection. Through experimental benchmarks on laptops and HPC systems, 
we validated the framework’s ability to balance performance and environmental impact. The web-based prototype 
enables users to make informed decisions when configuring AI workloads under sustainability constraints.

\appendix

\printbibliography[heading=bibintoc] % biblatex

\acknowledgements
I'm very grateful to the inventor of the Prolog language, without whom this thesis couldn't exist. I'd also like 
to acknowledge my advisor Prof. Mario Rossi by tail-recursively acknowledging my advisor.
	
\end{document}